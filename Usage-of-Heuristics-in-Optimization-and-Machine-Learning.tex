%%%%%%%%%%%%%%%%%%%%%%%%%%%%%%%%%%%%%%%%%
% Masters/Doctoral Thesis 
%
% Template license:
% CC BY-NC-SA 3.0 (http://creativecommons.org/licenses/by-nc-sa/3.0/)
%
%%%%%%%%%%%%%%%%%%%%%%%%%%%%%%%%%%%%%%%%%

% https://www.overleaf.com/latex/templates/template-for-a-masters-slash-doctoral-thesis/mkzrzktcbzfl#.Wj_WWFT1VPW

%----------------------------------------------------------------------------------------
%   Кофигуриране на пакети и опции.
%----------------------------------------------------------------------------------------

\documentclass[14pt,bulgarian,singlespacing,headsepline,oneside,openany]{MastersDoctoralThesis}

% Рефериране на думи и други обекти.
\usepackage{nameref}

% Използване на UNICODE.
\usepackage[utf8x]{inputenc}

% Кодировка за международни езици.
\usepackage[T1]{fontenc}

% ???
\usepackage{url}

% ???
\usepackage{lipsum}

% Използване на графика.
\usepackage{graphicx}

% Директория в която се намират изображенията.
\graphicspath{{images/}}

% ???
\usepackage{array}

% Автоматично създаване на различните индекси.
\usepackage{imakeidx}

% Добавя възможност за сензитивни хипер-връзки в самия документ.
% \usepackage{hyperref}

% ???
\usepackage{placeins}

% Автоматично създаване на азбучен указател.
\makeindex[columns=1, title=Азбучен указател, intoc]

%----------------------------------------------------------------------------------------
%   Настройки на страницата.
%----------------------------------------------------------------------------------------

\geometry{
	% Размер на листа.
	paper=a4paper,
	% Вътрешно отместване.
	inner=2.5cm,
	% Външно отместване.
	outer=3.8cm,
	% Отместване за свързване.
	bindingoffset=.5cm,
	% Отместване от горния край.
	top=1.5cm,
	% Отместване от долния край.
	bottom=1.5cm, % Bottom margin
	% Рамка на самата страница.
	%showframe,
}

%----------------------------------------------------------------------------------------
%   Данни за дисертацията.
%----------------------------------------------------------------------------------------

% Заглавие на дисертацията.
\thesistitle{ИЗПОЛЗВАНЕ НА ЕВРИСТИКИ ПРИ ОПТИМИЗАЦИЯ И МАШИННО САМООБУЧЕНИЕ}

% Вид на научната степен.
\degree{"Доктор на науките"} 

% Имена на автора.
\author{д-р инж. Тодор Димитров \textsc{Балабанов}}

% Служебен адрес.
\addresses{ИИКТ-БАН, ул. "акад. Георги Бончев", блок 2, етаж 5, кабинет 514, град София 1113, България} 

% Научна област.
\subject{01.01.12 "Информатика" \\ 4.6 "Информатика и компютърни науки" \\ 4 "Природни науки, математика и информатика"} 

% Ключови думи.
\keywords{} 

% Учебно заведение.
\university{\href{http://bas.bg}{Българската академия на науките}}

% Подразделение.
\faculty{\href{http://iict.bas.bg}{Институт по информационни и комуникационни технологии}} 

% Работна група.
\department{\href{http://iinf.bas.bg}{Моделиране и оптимизация}}

% Параметри на генерирания PDF файл.
\AtBeginDocument{
% Заглавие на PDF файла.
%	\hypersetup{pdftitle=\ttitle}
% Автор на PDF файла.
%	\hypersetup{pdfauthor=\authorname}
% Ключови думи на PDF файла.
%	\hypersetup{pdfkeywords=\keywordnames}
}

%----------------------------------------------------------------------------------------
%   Начало на същинския документ.
%----------------------------------------------------------------------------------------

\begin{document}

% Използване на римска номерация за страниците предхождащи същинското изложение.
\frontmatter

% Изчистване на стила за страницата преди да започне същинското изложение.
\pagestyle{plain}

%----------------------------------------------------------------------------------------
%   Заглавна страница.
%----------------------------------------------------------------------------------------

\begin{titlepage}
\begin{center}

% Лого на института.
\includegraphics[width=0.1\linewidth]{logo-iict-bg}

\vspace*{.06\textheight}

% Название на учебното заведение.
%{\scshape\LARGE \univname\par}\vspace{1.5cm}

% Работна група и отдел.
%\groupname\\\deptname\\[2cm]
 
% Автор.
%\authorname

% Хоризонтална линия.
\HRule \\[0.4cm]

% Заглавие на дисертацията.
{\huge \bfseries \ttitle\par}\vspace{0.4cm}

% Хоризонтална линия.
\HRule \\[1.5cm]

% Вид документ.
\textsc{\Large ДИСЕРТАЦИЯ}\\[0.5cm] 
 
\vfill

% Текст изискван от учебното заведение.
\large \textit{за присъждане на научна степен\\  \degreename}\\[0.3cm]

%TODO Област на дисертацията.
 
\vfill
 
% Година на предаване.
{\large София\\2017}
 
\vfill
\end{center}
\end{titlepage}

%----------------------------------------------------------------------------------------
%   Декларация за достоверност.
%----------------------------------------------------------------------------------------

\begin{declaration}
\end{declaration}

%----------------------------------------------------------------------------------------
%   Таблицата на съдържанието.
%----------------------------------------------------------------------------------------

\newpage
\tableofcontents

%----------------------------------------------------------------------------------------
%   Списък с фигурите.
%----------------------------------------------------------------------------------------

\newpage
\listoffigures

%----------------------------------------------------------------------------------------
%   Списък с таблиците. 
%----------------------------------------------------------------------------------------

\newpage
\listoftables

%----------------------------------------------------------------------------------------
%   Списък с абревиатурите.
%----------------------------------------------------------------------------------------

\begin{abbreviations}{ll}
\end{abbreviations}

%----------------------------------------------------------------------------------------
%   Основно изложение организирано в глави.
%----------------------------------------------------------------------------------------

% Номерация с арабски цифри за същинската част на дисертацията.
\mainmatter

% Оформление в стил на дисертация.
\pagestyle{thesis}

% По този начин номерацията на подточките е с арабски цифри.
\renewcommand\thesection{\thechapter.\arabic{section}}
\renewcommand\thesubsection{\thesection.\arabic{subsection}}

%\include{Chapters/Chapter1}
%\include{Chapters/Chapter2} 
%\include{Chapters/Chapter3}
%\include{Chapters/Chapter4} 
%\include{Chapters/Chapter5} 

%----------------------------------------------------------------------------------------
%   Приложения към основното изложени.
%----------------------------------------------------------------------------------------

\appendix

%\include{Appendices/AppendixA}
%\include{Appendices/AppendixB}
%\include{Appendices/AppendixC}

%----------------------------------------------------------------------------------------
%   Списък с използвана литература и източници на информация.
%----------------------------------------------------------------------------------------

\newpage
\begin{thebibliography}{99}
\end{thebibliography}

%----------------------------------------------------------------------------------------
%   Азбучен указател на използваните термини.
%----------------------------------------------------------------------------------------

\newpage
\printindex

\end{document}