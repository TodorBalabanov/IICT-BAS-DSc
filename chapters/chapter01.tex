Time-series Extreme Event Forecasting with Neural Networks at Uber

Предложение за иновативен модел на рекурентна изкуствена неверонна мрежа. Нуждата от подобрения идва от, това че LSTM не дава достатъчно приемливи резултати. Обучението на мрежата става с хетерогенен времеви ред. Целта на изследването е да прогнозира екстремални събития (дни и часове с голямо натоварване). 

Rough Neuron based RBF Neural Networks for Short-Term Load Forecasting

Оптимизация на параметрите в изкуствена невронна мрежа с използването на генетични алгоритми. Изкуствената невронна мрежа се използва за краткосрочна прогноза за енергийно натоварване (електрическа консумация). Представени са „груби“ (rough) неврони в радиално-базисна изкуствена невронна мрежа. Този тип неврони служат за подобряване на обобщаващите функции и за предпазване от преобучение (overfitting). Към всеки неврон във входния и скрития слой се добавят по два входа за „горна“ и „долна“ граница. 

An overview and comparative analysis of Recurrent Neural Networks for Short Term Load Forecasting

Споменават се двупосочно свързани мрежи, където прогнозата не зависи само от миналите периоди, но и от бъдещите периоди. Представен е и алгоритъмът за обучение с обратно разпространение на грешката във времето (Back Propagation Through Time) да „дълбоки“ мрежи. Представен е също метод за обучение на принципа на стохастично градиентно спускане (Stochastic Gradient Descent). Споменават се и мрежи с магистрална архитектура (Highway Network), където информацията преминава през група слоеве без да бъде мониторирана твърде много. Като инструмент за сравнените на прогнозиращи модели първоначално се използва времеви ред на Mackey-Glass. Едно от най-съществените заключения е, че по-ефективна прогноза се постига от по-сложните модели на изкуствени неверонни мрежи. 

